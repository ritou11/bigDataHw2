\documentclass[a4paper,12pt]{article}
\usepackage[noabs]{HaotianReport}

\title{第一次作业:QQ群组数据统计分析}
\author{刘昊天}
\authorinfo{电博181班, 2018310648}
\runninghead{大数据分析(B)课程报告}
\studytime{2018年10月-11月}

\renewcommand{\lstlistingname}{记录}
\crefname{listing}{记录}{记录}\Crefname{listing}{记录}{记录}
\graphicspath{{./}{./figure/}{./meta/fig/}}

\begin{document}
    \maketitle
    %\newpage
    \section{实验一:数据预处理}
    \paragraph{问题描述}
    将输入文件整理成唯独为用户*电影的矩阵$X$,其中$X(i,j)$为用户$i$对电影$j$的打分。输出两个矩阵:$X_{train}$和$X_{test}$,分别对应训练集和测试集。

    定义集合$U$为用户集合,共$N_u$个用户;集合$M$为电影集合,共$N_m$个电影。$X_{ij}$为用户$i\in U$对电影$j\in M$的评分。

    \section{实验二:协同过滤}
    \paragraph{问题描述}
    实现基于用户的协同过滤算法:猜测用户$i$是否喜欢电影$j$,只要看与$i$相似的用户是否喜欢$j$。与$i$越相似的用户,其对j的评分越有参考价值。
    \subsection{原理推导}
    根据原理写出
    $$
      \bar S_{ij} = \frac{\sum_{k\in U} Q_{ik}S_{kj}}{\sum_{k\in U} |Q_{ik}|}
    $$
    其中,$Q_{N_u\times N_u}$为相似度矩阵,$S_{N_u\times N_m}$为已知用户电影评分矩阵,$\bar S_{N_u\times N_m}$为估计用户电影评分矩阵。在本题中,$S=X_{train}$。

    \subsection{算法实现}

    \subsection{结果分析}

    \section{实验三:矩阵分解}
    \paragraph{问题描述}
    实现基于梯度下降的矩阵分解算法:将行为矩阵$X$分解为$U$和$V$两个矩阵的乘积,使$UV^T$在已知值部分逼近$X$。隐空间维度$k$是算法的参数,$U$和$V$可以认为是用户和电影在隐空间的特征表达,其乘积矩阵可预测$X$的未知部分。
    $$
      X_{N_u\times N_m} = U_{N_u\times k}V_{N_m\times k}^T
    $$
    \subsection{原理推导}
    根据题目提供的信息,目标函数如\cref{eq:exp3J}所示。本算法的核心就是通过迭代的方式,使得$J$最小,此时则认为$UV^T$对原矩阵$X$的拟合最佳。
    \begin{equation}
      \label{eq:exp3J}
      J =\frac{1}{2} ||A\circ (X-UV^T)||_F^2 + \lambda ||U||_F^2 + \lambda ||V||_F^2
    \end{equation}
    其中$<\circ>$运算符代表矩阵逐元素乘法。
    \subsection{算法实现}

    \subsection{结果分析}

    \label{applastpage}
    \newpage
    \bibliographystyle{ieeetr}
    \bibliography{report}
\iffalse
\begin{itemize}[noitemsep,topsep=0pt]
%no white space
\end{itemize}
\begin{enumerate}[label=\Roman{*}.,noitemsep,topsep=0pt]
%use upper case roman
\end{enumerate}
\begin{multicols}{2}
%two columns
\end{multicols}
\fi
\end{document}

\documentclass[a4paper,12pt]{article}
\usepackage[noabs]{HaotianReport}

\title{第二次作业:电影推荐}
\author{刘昊天}
\authorinfo{电博181班, 2018310648}
\runninghead{大数据分析(B)课程报告}
\studytime{2018年11月}

\graphicspath{{./}{./output/}}

\begin{document}
    \maketitle
    %\newpage
    \section{实验一:数据预处理}
    \paragraph{问题描述}
    将输入文件整理成唯独为用户*电影的矩阵$X$,其中$X(i,j)$为用户$i$对电影$j$的打分。输出两个矩阵:$X_{train}$和$X_{test}$,分别对应训练集和测试集。

    定义集合$U$为用户集合,共$N_u$个用户;集合$M$为电影集合,共$N_m$个电影。$X_{ij}$为用户$i\in U$对电影$j\in M$的评分。在本题中,$N_u=10000$,$N_m=10000$。

    在本题中,共提供了两组数据,其中训练集共6897746条记录,则$X_{train}$至多有$6.90\%$的非零元素;测试集共1719466条记录,$X_{test}$至多有$1.72\%$的非零元素。可见,$X_{train}$和$X_{test}$是十分稀疏的,因此需要用稀疏技术进行处理。

    在Python的scipy包中,有sparse模块包含了不同的稀疏矩阵类。其中csr\_matrix (Compressed Sparse Row marix) 为按行压缩的稀疏矩阵格式,另外还有csc(Compressed Sparse Column)、coo(COOrdinate)、bsr(Block Sparse Row)、dia(DIAgonal)、dok(Dictionary of Keys)、lil(List of Lists)等储存格式。不同的稀疏矩阵格式适用于不同的使用场景,各有优势劣势。csr\_matrix的优点在于高效的算术运算(CSR + CSR,CSR * CSR等)、行切片以及矩阵矢量积,而劣势是列切片及稀疏结构的变化。在本题中,数据矩阵不发生结构变化,且需参与大量运算,因此csr\_matrix是合适的。

    考查数据文件的格式发现,训练集、测试集的数据文件中用户uid是一个大整数,而我们需要将其映射到一个[0,9999]的区间中。这个映射是通过用户数据文件建立的,我们可以将用户出现在该文件中的行号减一作为用户在矩阵中的新id,记为$i$。由于数据文件行数很多,构建矩阵过程中需要大量查询用户的$i$值,因此映射$uid\rightarrow i$需要是高效的。从数据结构出发可以考虑使用HashMap,而在Python中只需要使用dict即可。建立该映射的代码如\cref{lst:readUserid}所示。电影mid在矩阵中的新id即为$j$,$j=mid-1$即可。
    \begin{lstlisting}[language=python,caption={readUserid},label=lst:readUserid]
def __init__(self, filename = ''):
    self.user_id = dict()
    if filename:
        self.readUserid(filename)
def readUserid(self, filename):
    with open(filename, 'r') as f:
        for i, l in enumerate(f):
            self.user_id[int(l)] = i
    \end{lstlisting}

    考虑到csr格式的劣势,我们不能采用逐个增加的方法构建稀疏矩阵,而应采用一次构建的方式。最终使用的生成代码如\cref{lst:getMatrixFromTxt}所示。
    \begin{lstlisting}[language=python,caption={getMatrixFromTxt},label=lst:getMatrixFromTxt]
# user index = row_number of user_id (from 0)
# movie index = movie_id - 1 (from 0)
def getMatrixFromTxt(self, filename, level=0):
    uid = list()
    mid = list()
    sc = list()
    with open(filename, 'r') as f:
        for i, l in enumerate(f):
            dt = l.split()
            score = int(dt[2]) - level
            if score != 0:
                uid.append(self.user_id[int(dt[0])])
                mid.append(int(dt[1]) - 1)
                sc.append(score)
    return csr_matrix((sc, (uid, mid)), shape=(hg.N, hg.N))
    \end{lstlisting}

    程序中,level参数表示构建矩阵时考虑的水平参数,读取的每个评分都将被减去level。从意义上将,在level代表了用户对没有打分的电影的评价,也即一个整体评价水平。这个参数是本报告引入的\textbf{关键参数},将极大影响最终结果。

    在MacBook Pro(3.1 GHz Intel Core i5, 16GB 2133 MHz LPDDR3, macOS Mojave 10.14.1)平台上进行测试,Python版本为3.6.2。生成矩阵耗时如\cref{tbl:exp1}所示,其中储存矩阵采用pickle包的dump函数进行储存。
    \begin{table}
      \centering
      \caption{生成数据矩阵耗时}
      \label{tbl:exp1}
      \begin{tabular}{ll}
        \toprule
        操作&用时\\
        \midrule
        构建测试矩阵&3514.88ms\\
        构建训练矩阵&13491.16ms\\
        储存矩阵&263.16ms\\
        \bottomrule
      \end{tabular}
    \end{table}
    \section{实验二:协同过滤}
    \paragraph{问题描述}
    实现基于用户的协同过滤算法:猜测用户$i$是否喜欢电影$j$,只要看与$i$相似的用户是否喜欢$j$。与$i$越相似的用户,其对j的评分越有参考价值。
    \subsection{原理推导}
    根据原理写出
    $$
      \bar S_{ij} = \frac{\sum_{k\in U} Q_{ik}S_{kj}}{\sum_{k\in U} |Q_{ik}|}
    $$
    其中,$Q_{N_u\times N_u}$为相似度矩阵,$S_{N_u\times N_m}$为已知用户电影评分矩阵,$\bar S_{N_u\times N_m}$为估计用户电影评分矩阵。在本题中,$S=X_{train}$。

    \subsection{算法实现}

    \subsection{结果分析}
    TODO: table for org,exp2,l3,lm
    在本任务中,首先考虑最简单的$level=0$情况,此时相当于认为用户对未评分的电影评分为0。
    \section{实验三:矩阵分解}
    \paragraph{问题描述}
    实现基于梯度下降的矩阵分解算法:将行为矩阵$X$分解为$U$和$V$两个矩阵的乘积,使$UV^T$在已知值部分逼近$X$。隐空间维度$k$是算法的参数,$U$和$V$可以认为是用户和电影在隐空间的特征表达,其乘积矩阵可预测$X$的未知部分。
    $$
      X_{N_u\times N_m} = U_{N_u\times k}V_{N_m\times k}^T
    $$
    \subsection{原理推导}
    根据题目提供的信息,目标函数如\cref{eq:exp3J}所示。本算法的核心就是通过迭代的方式,使得$J$最小,此时则认为$UV^T$对原矩阵$X$的拟合最佳。
    \begin{equation}
      \label{eq:exp3J}
      J =\frac{1}{2} ||A\circ (X-UV^T)||_F^2 + \lambda ||U||_F^2 + \lambda ||V||_F^2
    \end{equation}
    其中$<\circ>$运算符代表矩阵逐元素乘法。
    \subsection{算法实现}

    \subsection{结果分析}

    \label{applastpage}
    \newpage
    \bibliographystyle{ieeetr}
    \bibliography{report}
\iffalse
\begin{itemize}[noitemsep,topsep=0pt]
%no white space
\end{itemize}
\begin{enumerate}[label=\Roman{*}.,noitemsep,topsep=0pt]
%use upper case roman
\end{enumerate}
\begin{multicols}{2}
%two columns
\end{multicols}
\fi
\end{document}
